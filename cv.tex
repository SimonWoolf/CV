\documentclass[a4paper,10pt]{article}

\usepackage[top=3.0cm,right=3.0cm,bottom=3.0cm,left=3.0cm]{geometry}
\usepackage{url,parskip} 	%other packages for formatting
\RequirePackage{color,graphicx}
\usepackage[usenames,dvipsnames]{xcolor}
\usepackage{titlesec}					%custom \section
\usepackage{tabularx}

%Setup hyperref package, and colours for links
\usepackage{hyperref}
\definecolor{linkcolour}{rgb}{0,0.2,0.6}
\hypersetup{colorlinks,breaklinks,urlcolor=linkcolour, linkcolor=linkcolour}

%FONTS
\usepackage[urw-garamond]{mathdesign}

\titleformat{\section}{\Large\scshape\raggedright}{}{0em}{}[\titlerule]
\titlespacing{\section}{0pt}{15pt}{3pt}

\begin{document}
\pagestyle{empty}

\par{\centering
  {\Huge Simon \textsc{Woolf}
}\bigskip\par}

Young [programmer|maths geek|physics geek] doing high-quality freelance
full-stack web development. Experienced mostly with Ruby-based stacks (sinatra
\& rails), but very happy to work with any technology that'd be fun to learn!
Email me at \href{mailto:simon@simonwoolf.net}{simon@simonwoolf.net} to talk 
about anything :)

An up-to-date version of this CV is available at \href{https://github.com/SimonWoolf/CV}{https://github.com/SimonWoolf/CV}

\section{Links}

\begin{tabular}{rl}
  \textsc{website:}   & \href{http://simonwoolf.net}{http://simonwoolf.net} \\
  \textsc{email:}     & \href{mailto:simon@simonwoolf.net}{simon@simonwoolf.net} \\
  \textsc{twitter:}   & \href{https://twitter.com/SEMW}{https://twitter.com/SEMW} \\
\end{tabular}


\section{Selection of projects and freelance work}
\begin{tabularx}{\linewidth}{XXX} 
  \textsc{Disability Update} & \textsc{Sudoku} & \textsc{Websandbox} \\

  Website and case archive for Monika Sobiecki and Spencer Keen's Disability
  Discrimination Update, done on contract. Source publicly available by their
  kind permission. \href{https://github.com/SimonWoolf/disability-discrimination-update}{Source} &

  Sudoku webapp. Just for fun (done at Makers). Neat recursive solver/generator, guaranteed 
  to generate puzzles with a unique solution. \href{https://github.com/SimonWoolf/sudoku}{Source} &

  'Html-\&-css made fun' app. Everyone has their sandbox. Click an element for
  a pop-up that lets you edit its HTML and CSS. Fully versioned \& revertable. 
  Done at Makers. \href{https://github.com/SimonWoolf/websandbox}{Source} \\

  & & \\

  \textsc{Markov text generation} & \textsc{Data analytics with dc.js} &
  \textsc{Terminal-paint} \\

  Markov generation with an interactive twist - coming soon! &

  Client-side data processing with crossfilter and dc.js, for DirtJockey. 
  (Private client work, no source) &

  Basic terminal pixel painting program, done as a test at Makers to create a
  product from spec. \href{https://github.com/SimonWoolf/test6}{Source} \\
\end{tabularx}

\section{Languages \& skills}

\textsc{Ruby}---Multiple projects, including webapps in Rails and Sinatra. My current favourite language

\textsc{Python}---Various computational mathematics scripts and electromagnetic field lines visualisation project, as part of my degree

\textsc{Unix}---Used Linux in various forms since uni, mostly debian-based
distros. Poor to middling bash scripter---I tend to go to ruby or python for any more than trivial scripts

\textsc{Miscellanious webdev}---javascript (\& jQuery ), HTML / CSS. Tend to like using preprocessors for most of those (coffeescript, haml, sass/scss, and the like) for big rails projects

\textsc{Maths \& Physics}---In addition to my degree, have tutored A-level maths and physics in my spare time for a few years now

\textsc{Miscellanious}---LaTeX. Vimmer (vimist? vimian?), but haven't yet got up the courage to properly learn vimscript. Dabbled in Java to make an Android app I wanted, but wasn't a big fan

Of course, very happy to learn new things not in this list :)



\section{Education}
\begin{tabularx}{\linewidth}{>{\raggedleft\hsize=0.42\hsize}X>{\hsize=1.5\hsize}X} 
  \textsc{2006-2010} & BA MSci in \textsc{Mathematics and Physics}, Cambridge University\\
  & \footnotesize{Took the three year Mathematics BA (in first year, also took the Physics part 
  of the NatSci tripos), followed by the fourth year of the NatSci Physics MSci.} \\
  \multicolumn{2}{c}{} \\
  \textsc{2011-2013} & \textsc{Law LLB}, Queen Mary, University of
  London\\
  & \footnotesize{Sidetrack into a Law conversion course. Got excellent results (and was a 
    quarter-finalist in the ESU national moot), but luckily realised that I wanted to be a 
  programmer rather than a lawyer before going too far down that road.} \\
  \multicolumn{2}{c}{} \\
  \textsc{2013-2014} & Makers Academy \\
  & \footnotesize{Intensive twelve-week programming \& web development course.
    Teaches programming best practices (TDD, software design, version control,
    agile), using ruby with sinatra, rails, HTML and CSS, javascript \& jQuery,
    etc. Took it to get a firm grounding of programming \& webdev essentials, and
  get rid of all the bad habits I'd built up doing scientific programming...} \\
  \multicolumn{2}{c}{} \\
  \textsc{School} & \footnotesize{A levels (Double Maths, Physics, English Lit.): AAAB respectively} \\
  & \footnotesize{STEP Mathematics: 1,1} \\
  & \footnotesize{AEA Physics: Merit} \\
  & \footnotesize{GCSEs: 7 A*s, 3 As}
\end{tabularx}

\end{document}
